\documentclass[conference]{IEEEtran}
\IEEEoverridecommandlockouts
% The preceding line is only needed to identify funding in the first footnote. If that is unneeded, please comment it out.
\usepackage{cite}
\usepackage{amsmath,amssymb,amsfonts}
\usepackage{algorithmic}
\usepackage{graphicx}
\usepackage{textcomp}
\usepackage{xcolor}

\usepackage[utf8]{vietnam}

\usepackage{amsthm}
% \usepackage{ntheorem}
\newtheoremstyle{theoremst}% name of the style to be used
  {\topsep}% measure of space to leave above the theorem. E.g.: 3pt
  {\topsep}% measure of space to leave below the theorem. E.g.: 3pt
  {\normalfont}% name of font to use in the body of the theorem
  {0pt}% measure of space to indent
  {\bfseries}% name of head font
  {.}% punctuation between head and body
  { }% space after theorem head; " " = normal interword space
  {\thmname{#1}\thmnumber{ #2}\textnormal{\thmnote{ (#3)}}}

\newtheoremstyle{examplest}% name of the style to be used
  {\topsep}% measure of space to leave above the theorem. E.g.: 3pt
  {\topsep}% measure of space to leave below the theorem. E.g.: 3pt
  {\normalfont}% name of font to use in the body of the theorem
  {0pt}% measure of space to indent
  {\bfseries}% name of head font
  {\\}% punctuation between head and body
  { }% space after theorem head; " " = normal interword space
  {\thmname{#1}\thmnumber{ #2}\textnormal{\thmnote{ (#3)}}}
  
\theoremstyle{theoremst}
\newtheorem{definition}{Định nghĩa}[section]
\newtheorem{theorem}{Định lý}[section]


\def\BibTeX{{\rm B\kern-.05em{\sc i\kern-.025em b}\kern-.08em
    T\kern-.1667em\lower.7ex\hbox{E}\kern-.125emX}}
\begin{document}


\title{MAGNN: Metapath Aggregated Graph Neural Network for
Heterogeneous Graph Embedding\\
{\footnotesize \textsuperscript{}Giảng viên hướng dẫn: TS. Đỗ Thị Thanh Hà}
\thanks{Identify applicable funding agency here. If none, delete this.}
}

\author{\IEEEauthorblockN{ Nguyễn Mạnh Linh}
% \IEEEauthorblockA{\textit{dept. name of organization (of Aff.)} \\
% \textit{name of organization (of Aff.)}\\
% City, Country \\
% email address or ORCID}
\and
\IEEEauthorblockN{Nguyễn Đức Thịnh}
% \IEEEauthorblockA{\textit{dept. name of organization (of Aff.)} \\
% \textit{name of organization (of Aff.)}\\
% City, Country \\
% email address or ORCID}
% \and
% \IEEEauthorblockN{3\textsuperscript{rd} Given Name Surname}
% \IEEEauthorblockA{\textit{dept. name of organization (of Aff.)} \\
% \textit{name of organization (of Aff.)}\\
% City, Country \\
% email address or ORCID}
% \and
% \IEEEauthorblockN{4\textsuperscript{th} Given Name Surname}
% \IEEEauthorblockA{\textit{dept. name of organization (of Aff.)} \\
% \textit{name of organization (of Aff.)}\\
% City, Country \\
% email address or ORCID}
% \and
% \IEEEauthorblockN{5\textsuperscript{th} Given Name Surname}
% \IEEEauthorblockA{\textit{dept. name of organization (of Aff.)} \\
% \textit{name of organization (of Aff.)}\\
% City, Country \\
% email address or ORCID}
% \and
% \IEEEauthorblockN{6\textsuperscript{th} Given Name Surname}
% \IEEEauthorblockA{\textit{dept. name of organization (of Aff.)} \\
% \textit{name of organization (of Aff.)}\\
% City, Country \\
% email address or ORCID}
}

\maketitle

\begin{abstract}
Một lượng lớn các đồ thị hay mạng trong thực tế vốn dĩ không đồng nhất, có nhiều loại nút và nhiều loại quan hệ. Embedding đồ thị không đồng nhất là việc embed từ cấu trúc lớn và nhiều thông tin của đồ thị về biểu diễn nút trong không gian thấp chiều. Các mô hình đã tồn tại tường định nghĩa metapaths trong một đồ thị không đầu nhất để ghi lại các quan hệ và định hướng lựa chọn "hàng xóm". Tuy nhiên các mô hình này bỏ qua đặc trưng của từng nút mà tìm hiểu ngay lập tức các nút trên metapath hoặc chỉ xem xét một metapath. Để khắc phục ba giới hạn này, tác giả đề xuất một mô hình mới là \textit{Metapath Aggregated
Graph Neural Network} (MAGNN) để tăng tốc hiệu năng cuối cùng. Đặc biệt, MAGNN sử dụng ba thành phần chính, biến đổi nội dung của nút thành các thuộc tính đóng gói của nút đầu vào, tổng hợp intra-metapath để kết hợp các nút ngữ nghĩa trung gian và tổng hợp inter-metapath để kết hợp thông tin từ nhiều metapaths. Các thí nghiệm được thực hiện trên ba bộ dữ liệu đồ thị không đồng nhất trong thực tế để phân loại nút, phân cụm nút và dự đoán liên kết chỉ ra rằng MAGNN đạt được kết quả dự đoán chính xác hơn so với các mô hình state-of-the-art hiện tại .
\end{abstract}

% \begin{IEEEkeywords}
% component, formatting, style, styling, insert
% \end{IEEEkeywords}

\section{Giới thiệu}
Nhiều bộ dữ liệu thực tế được biểu diễn với cấu trúc dữ liệu đồ thị, trong đó các đối tượng và quan hệ giữa chúng được biểu diễn bằng các nút và cạnh. Các ví dụ bao gồm mạng xã hội [14, 29], hệ thống vật lý [2, 10], mạng giao thông [18, 34], mạng trích dẫn [1, 14, 16], hệ thống gợi ý [26, 35], đồ thị tri thức [3, 24], ... Bản chất non-Euclidean của đồ thị khiến chúng khó mô hình hóa bằng các mô hình học máy truyền thống. Với tập hàng xóm của mỗi nút, không hề có thứ tự hoặc giới hạn về kích thước, Tuy nhiên, hầu hết các mô hình thống kê giả định rằng một đầu vào có thứ tự và kích thước cố định trong không gian Euclid. Do đó, sẽ thuận tiện nếu các nút có thể được biểu diễn bằng các vector thấp chiều trong không gian Euclid và từ đó có thể lấy làm đầu vào của mô hình học maý khác. 

Các kĩ thuật embed đồ thị khác nhau được đề xuất cho cấu trúc dữ liệu đồ thị. LINE [25]

\section{Sơ lược}
Trong phần này, tác giả đưa ra các định nghĩa chuẩn của một số thuật ngữ quan trọng liên quan đến đồ thị không thuần nhất. Minh họa trong hình 1. Bên cạnh đó bảng 1 tóm tắt các kí hiệu được sử dụng nhiều trong báo cáo để thuận tiện cho việc tra cứu nhanh.

\begin{definition}
\textbf{Đồ thị không đồng nhất.} Một đồ thị không đồng nhất được định nghĩa là một đồ thị $\pmb{\mathcal{G}} = (\pmb{\mathcal{V}}, \pmb{\mathcal{E}})$ với ánh của loại nút $\phi: \pmb{\mathcal{V}} \to \pmb{\mathcal{A}}$ và ánh xạ của loại cạnh $\psi: \pmb{\mathcal{E}} \to \pmb{\mathcal{R}}$. $\pmb{\mathcal{A}}$ và $\pmb{\mathcal{R}}$ lần lượt là các tập loại nút và loại cạnh với $|\pmb{\mathcal{A}}| + |\pmb{\mathcal{R}}| > 2$.
\end{definition}

\begin{definition}
  \textbf{Metapath.} Một metapath $P$ được định nghĩa là một đường đi lập thành từ $A_1  \xrightarrow{R_1} A_2  \xrightarrow{R_2} ... \xrightarrow{R_l} A_{l+1}$ (viết tắt là $A_1 A_2 ... A_{l+1}$) mô tả một quan hệ tổng hợp $R = R_1 \circ R_2 \circ ... \circ R_l$ giữa các loại nút $A_1$ và $A_{l+1}$, trong đó $\circ$ là toán tử tổng hợp trên các quan hệ.
\end{definition}

\begin{definition}
  \textbf{Cấu hình metapath.} Cho một metapath $P$ của một đồ thị không đồng nhất, một cấu hình metapath $p$ của $P$ được định nghĩa là một dãy các nút trong đồ thị theo lược đồ được xác định bởi $P$.
\end{definition}

\begin{definition}
  \textbf{Lân cận dựa trên metaptah.} Cho một metapath $P$ của một đồ thị không đồng nhất, các lân cận dựa trên metapath $\pmb{\mathcal{N}}_\upsilon ^ P$ của một nút $\upsilon$ được định nghĩa là tập hợp các nút liên kết với nút $\upsilon$ qua các cấu hình metapath của $P$. Một lân cận được kết nối bởi hai cấu hình metapath khác nhau được đánh giá như hai nút khác nhau trong $\pmb{\mathcal{N}}_\upsilon ^ P$. Lưu ý rằng $\pmb{\mathcal{N}}_\upsilon ^ P$ bao gồm chính nút $\upsilon$ nếu $P$ đối xứng.

  Ví dụ, xem xét tập dữ liệu UATA trong hình 1, nghệ sĩ \textit{Queen} là một lân cân dựa trên metapath của người dùng \textit{Bob}. Hai nút này được kết nối thông qua cấu hình metapath \textit{Bob-Beatles-Rock-Queen}. Hơn nữa, chúng ta có thể tham chiếu tới \textit{Beatles} và \textit{Rock} như là các nút trung gian trên cấu hình metapath này.
\end{definition}

\begin{definition}
  \textbf{Đồ thị dựa trên metapath.} Cho một metapath $P$ của một đồ thị không đồng nhất $\pmb{\mathcal{G}}$, đồ thị dựa trên metapath $\pmb{\mathcal{G}}^P$ là một đồ thị được xây dựng bởi tất cả các cặp lân cận dựa trên metapath $P$ trong đồ thị $\pmb{\mathcal{G}}$. Lưu ý rằng $\pmb{\mathcal{G}}^P$ là đồng nhất nếu $P$ đối xứng.
\end{definition}

\begin{definition}
  \textbf{Biểu diễn đồ thị không đồng nhất.} Cho một đồ thị không đồng nhất $\pmb{\mathcal{G}} = (\pmb{\mathcal{V}}, \pmb{\mathcal{E}})$ với các ma trận thuộc tính nút $\pmb{X}_{A_i} \in \mathbb{R} ^ {|\pmb{\mathcal{V}}_{A_i}| \times d_{A_i}}$ của các loại nút $A_i \in \pmb{\mathcal{A}}$, biểu diễn đồ thị không đồng nhất là việc học các biểu diễn nút $d$ chiều $\pmb{h}_{\upsilon} \in \mathbb{R}^d$ với mọi $\upsilon \in \pmb{\mathcal{V}}$ với $d \ll |\pmb{\mathcal{V}}|$ có thể ghi lại thông tin cấu trúc và ngữ nghĩa liên quan đến $\pmb{\mathcal{G}}$.
\end{definition}

\begin{figure*}
  \includegraphics[width=\textwidth]{figs/fig1.png}
  \caption{Minh họa các thuật ngữ được định nghĩa trong Phần 2. (a) Một ví dụ về đồ thị không đồng nhất với ba loại nút (người dùng, nghệ sĩ, thẻ). (b) metapath Người dùng-Nghệ sĩ-Thẻ-Nghệ sĩ (UATA) và metapath Người dùng-Nghệ sĩ-Thẻ-Nghệ sĩ-Người dùng (UATAU). (c) Ví dụ các cấu hình metapath của UATA, UATAU. (d) Đồ thị dựa trên metapath cho UATA và UATAU.}
\end{figure*}





\section{Nghiên cứu liên quan}
Trong phần này, tác giả xem lại các nghiên cứu về học biểu diễn đồ thị liên quan tới mô hình của tác giả. Chúng được chia thành hai phần nhỏ: phần đầu tóm tắt các nỗ lực nghiên cứu về GNN cho việc nhúng đồ thị, phần sau giới thiệu các phương pháp nhúng đồ thị cho các đồ thị không đông nhất.

\subsection{Mạng neuron đồ thị}
Mục tiêu của một GNN là học một biểu diễn vector thấp chiều $\pmb{h}_{\upsilon}$ cho mọi nút $\upsilon$, cách được sử dụng cho nhiều tác vụ downstream, ví dụ như phân loại nút, phân cụm nút và dự đoán liên kết. Lý do đằng sau điều này là mỗi nút được xác định một cách tự nhiên bởi các đặc trưng và khu vực lân cận của nó. Theo ý tưởng này và dựa trên quá trình xử lý tín hiệu đồ thị, GNNs dựa trên phổ được phát triển để thực hiện tích chập đồ thị trong miền Fourier của một đồ thị. ChebNet [8] sử dụng các đa thức Chebusev để lọc các tín hiệu đồ thị (đặc trưng nút) trong miền Fourier của đồ thị. Một mô hình có ảnh hưởng khác thuộc loại này là GCN [16], nó ràng buộc và đơn giản hóa các tham số của ChebNet để giảm bớt vấn đề overfitting và cải thiện hiệu năng. Tuy nhiên, GNNs dựa trên phổ có khả năng mở rộng và tổng quát kém vì chúng yêu cầu toàn bộ đồ thị làm đầu vào cho mọi lớp và các bộ lọc đã được học của chúng phụ thuộc vào cơ sở riêng của Laplacian của đồ thị, liên quan chặt chẽ đến cấu trúc đồ thị cụ thể.

\section{Phương pháp}
Trong phần này, tác giả mô tả một mạng neuron đồ thị tổng hợp metapath mới (MAGNN) để nhúng đồ thị không đồng nhất. MAGNN được xây dựng bởi 3 thành phần chính: biến đổi nội dung nút, tổng hợp hợp intra-metapath và tổng hợp inter-metapath. Hình 2 minh họa việc tạo nhúng của một nút. Các quá trình lan truyền tiến được chỉ ra trong thuật toán 1.

\subsection{Biến đổi nội dung nút}
Với một đồ thị không đồng nhất liên kết với các thuộc tính nút, các loại nts khác nhau có thể có chiều của các vector đặc trưng không bằng nhau. Kể cả chúng có số chiều bằng nhau thì chúng cũng nằm trên các không gian đặc trưng khác nhau. Ví dụ các bag-of-words vectors $n_1$ chiều của đoạn văn bản và các vectors biểu đồ cường độ $n_2$ chiều của hình ảnh không thể  trực tiếp hoạt động cùng nhau kể cả $n_1 = n_2$. Các vectors đặc trưng với các chiều khác nhau là một khó khăn khi tác giả xử lý chúng trong một framework thống nhất. Do đó, tác giả cần chieus các loại khác nhau của đặc trưng nút vào cùng một không gian vector latent trước.

Vì vậy trước khi đưa các vectors nút vào MAGNN, tác giả áp dụng phép biến đổi tuyến tính cho từng loại nút bằng cách chiếu các vector đặc trưng vào cùng một không gian latent. Với một nút $\nu \in \pmb{\mathcal{V}}_A$ của loại $A \in \pmb{\mathcal{A}}$, ta có
\begin{equation}
  \mathbf{h'}_{\nu} = \mathbf{W}_A \cdot \mathbf{x}^A_{\nu}
\end{equation}
trong đó $\mathbf{x}_{\nu} \in \mathbb{R}^{d_A}$ là vector đặc trưng gốc và $\mathbf{h'}_{\nu} \in \mathbb{R}^{d'}$ là vector lantent hình chiếu  của nút $\nu$. $\mathbf{W}_A \in \mathbb{R}^{d' \times d_A}$ là ma trận trọng số của các nút loại $A$.

Biến đổi nội dung nút giải quyết tính không đồng nhất của một đồ thị bắt nguồn từ các đặc trưng nội dung nút. Sau khi áp dụng tác động này, tất cả các đặc trưng chiếu của nút đều có cùng chiều, tạo điều kiện thuận lợi cho quá trình tổng hợp của thành phần tiếp theo của mô hình. 

\subsection{Tổng hợp intra-metapath}
Cho một metapath $P$, lớp tổng hợp intra-metapath học thông tin cấu trúc và ngữ nghĩa được nhúng trong nút mục tiêu, các lân cận dựa trên metapath và ngữ cảnh ở giữa bằng cách mã hóa cấu hình metapath của $P$. Gọi $P(\nu, u)$ là một cấu hình metapath kết nối nút mục tiêu $\nu$ và lân cận dựa trên metapath $u \in \pmb{\mathcal{N}}^P_{\nu}$, tác giả định nghĩa thêm  nút trung gian của $P(\nu, u)$ như sau $\{ m^{P(\nu, u)} \} = P(\nu, u) \backslash \{ \nu, u \}$. Tổng hợp intra-metapath sử dụng một bộ mã hóa cấu hình metapath để biến đổi tất cả các đặc trưng nút dọc theo một cấu hình metapath thành một vector duy nhất,
\begin{equation}
  \mathbf{h}_{P(\nu, u)} = f_{\theta} (P(\nu, u)) = f_{\theta} \left( \mathbf{h'}_{\nu}, \mathbf{h'}_{u}, \{ \mathbf{h'}_{t}, \forall t \in \{m^{P(\nu, u)}\} \} \right)
\end{equation}
trong đó, $\mathbf{h}_{P(\nu, u)} \in \mathbb{R}^{d'}$ có số chiều là $d'$. Để đơn giản, ta dùng $P(\nu, u)$ để biểu diễn một cấu hình đơn, mặc dù có thể có nhiều cấu hình kết nối 2 nút. Phần sau sẽ giới thiệu một vài lựa chọn của bộ mã hóa cấu hình metapath tốt.

Sau khi mã hóa các cấu hình metapath thành biểu diễn vector, tác giả áp dụng một lớp chú ý đồ thị [28] để  tính tổng trọng số các cấu hình metapath của $P$ liên quan đến nút đích $\nu$. Ý tưởng chính là các cấu hình metapath khác nhau sẽ đóng góp vào biểu diễn nút mục tiêu với mức độ khác nhau. Chúng ta có thể mô hình hóa điều này bằng cách học một trọng số chuẩn hóa quan trọng $\alpha^P_{\nu u}$ cho mỗi cấu hình metapath và sau đó tính tổng trọng số của tất cả các cấu hình:
\begin{equation}
  \begin{split}
    e^P_{\nu u} &= \text{LeakyReLU} (a^T_P \cdot [\mathbf{h'}_{\nu}\parallel \mathbf{h}_{P(\nu, u)}]), \\
  \alpha ^P_{\nu u} &= \frac{\text{exp}{e^P_{\nu u}}}{\sum _{s \in \pmb{\mathcal{N}}^P_{\nu}} \text{exp} (e^P_{\nu u})}, \\
  \mathbf{h}^P_{\nu} &= \sigma \left( \sum_{u \in \pmb{\mathcal{N}}^P_{\nu}} \alpha ^P_{\nu u} \cdot \mathbf{h}_{P(\nu, u)} \right).
  \end{split}
\end{equation}
Trong đó, $a_P \in \mathbb{R}^{2d'}$ là vector chú ý được tham số hóa cho metapath $P$ và $\parallel$ kí hiệu cho toán tử nối vector. $e^P_{\nu u}$ chỉ độ quan trọng của cấu hình metapath $P(\nu, u)$ đến nút $\nu$, nút sau đó được chuẩn hóa theo các lựa chọn $u \in \pmb{\mathcal{N}}^P_{\nu}$ sử dụng hàm softmax. Do trọng số chuẩn hóa $\alpha ^P_{\nu u}$ được lấy cho tất cả $u \in \pmb{\mathcal{N}}^P_{\nu}$, chúng được sử dụng để tính toán một tổ hợp có trọng số của các biểu diễn của các cấu hình metapath cho nút $\nu$. Cuối cùng, đầu ra chạy qua hàm kích hoạt $\sigma (\cdot)$.

Cơ chế chú ý cũng có thể được mở rộng thành nhiều nhánh, điều giúp ổn định quá trình học và giảm đi phương sai lớn từ tính không đồng nhất của đồ thị. Nghĩa là, chúng ta thực hiện $K$ cơ chế chú ý độc lập và sau đó nối đầu ra của chúng lại, kết quả thu được trong biểu thức sau:
\begin{equation}
  \mathbf{h}^P_{\nu} = \parallel ^K_{k=1} \sigma \left( \sum_{u \in \pmb{\mathcal{N}}^P_{\nu}} \left[ \alpha ^P_{\nu u} \right]_k \cdot \mathbf{h}_{P(\nu, u)} \right)
\end{equation} 
trong đó $\left[ \alpha ^P_{\nu u} \right]_k$ là trọng số chuẩn hóa của cấu hình metapath $P(\nu, u)$ đến nút $\nu$ tại nhánh chú ý thứ $k$.

Tóm lại, với các vector đặc trưng $\mathbf{h'}_{u} \in \mathbb{R}^{d'} \forall u \in \pmb{\mathcal{V}}$ và tập các metapaths $\pmb{\mathcal{P}}_A = {P_1, P_2, ..., P_M}$ bắt đầu hoặc kết thúc với loại nút $A \in \pmb{\mathcal{A}}$, tổng hợp của MAGNN sinh $M$ biểu diễn metapath-specific vector của nút đích $\nu \in \pmb{\mathcal{V}}_A$, kí hiệu là $\{ \mathbf{h}^{P_1}_{\nu}, \mathbf{h}^{P_2}_{\nu}, ... , \mathbf{h}^{P_M}_{\nu} \}$. Mỗi $\mathbf{h}^{P_1}_{\nu} \in \mathbb{R}^{d'}$ (giả sử $K=1$) có thể hiểu là tổng hợp của các cấu hình $P_i-\text{metapath}$ của nút $\nu$, thể hiện một khía cạnh của thông tin chứ chong nút $\nu$.

\subsection{Tổng hợp inter-metapath}
SAu khi tổng hợp dữ liệu nút và cạnh với mỗi metapath, chúng ta cần kết hợp thông tin của tất cả các metapath sử dụng một lớp tổng hợp inter-metapath. Bây giờ với một loại nút $A$, ta có $|\pmb{\mathcal{V}}_A|$ tập các latent vectors: $\{ \mathbf{h}^{P_1}_{\nu}, \mathbf{h}^{P_2}_{\nu}, ... , \mathbf{h}^{P_M}_{\nu} \}$ với $\nu \in \pmb{\mathcal{V}}_A$, với $M$ là số metapaths cho loại $A$. Một các tiếp cận tổng hợp inter-metapath trực tiếp là lấy trung bình element-wise của các vectors nút này. Ta mở rộng cách tiếp cận này bằng cách khai thác cơ chế chú ý để gán các trọng số khác nhau cho các metapaths khác nhau. Phép toán này là có lý vì các metapaths có đóng góp không giống nhau trong một đồ thị không đồng nhất.

Đầu tiên, ta cộng mỗi metapath $P_i \in \pmb{\mathcal{P}}_A$ bằng trung bình các vectors nút metapath-specific đã được biến đổi cho tất cả các nút $\nu \in \pmb{\mathcal{V}}_A$,
\begin{equation}
  \mathbf{s}_{P_i} = \frac{1}{|\pmb{\mathcal{V}}_A|} \sum_{\nu \in \pmb{\mathcal{V}}_A} \text{tanh} \left( \mathbf{M}_A \cdot \mathbf{h}^{P_i}_{\nu} + \mathbf{b}_A \right)
\end{equation}
trong đó, $\mathbf{M}_A \in \mathbb{R}^{d_m \times d'}$ và $\mathbf{b}_A \in \mathbb{R}^{d_m}$ là các tham số học.

Sau đó ta sử dụng cơ chế chú ý để  hợp nhất các vectors nút metapath-specific của $\nu$ như sau:
\begin{equation}
  \begin{split}
    e_{P_i} &= \mathbf{q}^T_A \cdot \mathbf{s}_{P_i}, \\
    \beta _{P_i} &= \frac{\text{exp}(e_{P_i})}{\sum_{P \in \pmb{\mathcal{P}}_A} \text{exp}(e_{P})}, \\
    \mathbf{h}^{\pmb{\mathcal{P}}_A}_{\nu} &= \sum_{P \in \pmb{\mathcal{P}}_A} \beta _{P} \cdot \mathbf{h}^P_{\nu}
  \end{split}
\end{equation}
trong đó, $\mathbf{q}_A \in \mathbb{R}^{d_m}$ laf vector chú ý tham số hóa cho loại nút $A$. $\beta _{P_i}$ có thể hiểu là độ đóng góp tương đối của metapath $P_i$ cho các nút loại $A$. Do $\beta _{P_i}$ được tính toán cho mỗi $P_i \in \pmb{\mathcal{P}}_A$, ta có thể tính tổng có trọng số tất cả các vectors nút metapath-specific của $\nu$.

Cuối cùng, MAGNN sử dụng một biến đổi tuyến tính bổ sung với một hàm phi tuyến để chiếu các nút nhúng vào không gian vector với số chiều đầu ra mong muốn:
\begin{equation}
  \mathbf{h}_{\nu} = \sigma \left( \mathbf{W}_0 \cdot \mathbf{h}^{\pmb{\mathcal{P}}_A}_{\nu} \right)
\end{equation}
trong đó, $\sigma (\cdot)$ là một hàm kích hoạt và $\mathbf{W}_0 \in \mathbb{R}^{d_0 \times d'}$ là một ma trận trọng số. Phép chiếu này là một tác vụ cụ thể. Nó có thể hiểu là một phân loại tuyến tính cho phân loại nút hoặc được coi là hình chiếu vào không gian với các độ đo sự tương tự của nút cho dự đoán liên kết. 

\section{Thực nghiệm}
Trong phần này, chúng tôi trình bày các thực nghiệm để chứng minh tính hiệu quả của MAGNN đối với việc biểu diễn đồ thị không đồng nhất. Các thực nghiệm nhằm giải quyết các câu hỏi nghiên cứu sau:

\begin{itemize}
  \item RQ1. MAGNN có hiệu quả như thế nào trong việc phân loại các nút?
  \item RQ2. MAGNN có hiệu quả như thế nào trong việc phân cụm các nút?
  \item RQ3. MAGNN có hiệu quả như thế nào trong việc dữ đoán các liên kết hợp lí giữa các cặp nút?
  \item RQ4. Ảnh hưởng của 3 thành phần chính của MAGNN đã được mô tả trong các phần trước đó là gì?
  \item RQ5. Làm cách nào để ta có thể xác định được tính đại diện của những phương pháp biểu diễn đồ thị khác nhau?
\end{itemize}

\subsection{Tập dữ liệu}
Chúng tôi lựa chọn 03 tập dữ liệu đồ thị không đồng nhất phổ biến nhất hiện nay từ các lĩnh vực khác nhau để đánh giá hiệu năng của MAGNN khi so sánh với các phương pháp cơ sở là các phương pháp tối ưu nhất hiện nay (baselines). Cụ thể, tập dữ liệu IMDb và DBLP được sử dụng trong thực nghiệm liên quan đến việc phân loại nút và phân cụm nút. Tập dữ liệu Last.fm được sử dụng cho thực nghiệm về khả năng dự báo mối quan hệ. Những giá trị thống kê cơ bản của 03 tập dữ liệu được tóm tắt trong Bảng 2, và mô hình mạng được thể hiện trong Hình 3. Chúng tôi sử dụng vector one-hot cho các nút không có thuộc tính như là các thuộc tính đầu vào giả (dummy) của chúng. 

\begin{itemize}
  \item $\mathbf{IMDb}^{1}$ là cơ sở dữ liệu trực tuyến về các bộ phim và chương trình truyền hình, bao gồm các thông tin như dàn diễn viên, đội ngũ sản xuất và tóm tắt cốt truyện. Chúng tôi sử dụng một tập mẫu được lấy ra từ IMDb, thông qua quá trình tiền xử lí dữ liệu thu được 4278 bộ phim, 2081 đạo diễn và 5257 diễn viên. Phim được gắn nhãn là một trong ba loại (Hành động, Hài kịch và Chính kịch) dựa trên thông tin thể loại của chúng. Mỗi bộ phim cũng được mô tả bằng một bag-of-words đại diện cho các từ khóa cốt truyện của chúng. Đối với các mô hình học bán giám sát, các nút phim được chia thành các tập huấn luyện, xác thực và kiểm tra với kích thước lần lượt là 400 (9,35\%), 400 (9,35\%) và $3478(81,30 \%)$ nút.
  \item $\mathbf{DBLPP}^{2}$ là một trang web tổng hợp danh mục tài liệu về khoa học máy tính. Chúng tôi sử dụng một tập mẫu được lấy ra từ DBLP [12, 15], sau khi tiền xử lí dữ liệu thu được thông tin về 4057 tác giả, 14328 bài báo, 7723 thuật ngữ và 20 nơi xuất bản. Các tác giả được chia thành bốn lĩnh vực nghiên cứu (Cơ sở dữ liệu, Khai thác dữ liệu, Trí tuệ nhân tạo và Truy xuất thông tin). Mỗi tác giả được mô tả bằng một bag-of-words đại diện cho các từ khóa trong bài báo của họ. Đối với các mô hình học bán giám sát, các nút tác giả được chia thành các tập huấn luyện, xác thực và kiểm tra với kích thước lần lượt là $400(9,86 \%)$ 400 (9,86\%) và 3257 (80,28\%) nút.
  \item $\mathbf{Last.fm}^{3}$ là trang web âm nhạc theo dõi thông tin hành vi nghe nhạc của người dùng từ nhiều nguồn khác nhau. Chúng tôi sử dụng bộ dữ liệu do HetRec 2011 phát hành [4], sau khi tiền xử lí dữ liệu thu được thông tin về 1892 người dùng, 17632 nghệ sĩ và 1088 thẻ nghệ sĩ. Tập dữ liệu này được sử dụng cho tác vụ dự đoán liên kết giữa các nút, trong đó tập dữ liệu không chưa bất cứ thông tin nào liên quan đến nhãn hay các đặc điểm đặc trưng của đối tượng. Đối với các mô hình học bán giám sát, các cặp người dùng-nghệ sĩ được chia thành các tập huấn luyện, xác thực và kiểm tra với kích thước lần lượt là $64984(70 \%), 9283(10 \%)$ và $18567(20 \%)$ cặp.
\end{itemize}

\subsection{Mô hình cơ sở để tham chiếu}
Chúng tôi so sánh MAGNN với nhiều loại mô hình biểu diễn đồ thị khác nhau, bao gồm các mô hình biểu diễn đồ thị đồng nhất truyền thống (trái ngược với GNN), mô hình biểu diễn đồ thị không đồng nhất truyền thống, GNN cho đồ thị đồng nhất và GNN cho đồ thị không đồng nhất. Ta gọi chúng lần lượt là mô hình đồng nhất truyền thống, mô hình không đồng nhất truyền thống, GNN đồng nhất và GNN không đồng nhất. Danh sách các mô hình cơ sở được thể hiện dưới đây.

\begin{itemize}
  \item $\mathbf{LINE}$ [25] là một mô hình đồng nhất truyền thống khai thác mức độ tương đồng bậc nhất và bậc hai giữa các nút. Chúng tôi áp dụng mô hình này cho các đồ thị không đồng nhất bằng cách bỏ qua tính không đồng nhất của cấu trúc đồ thị và loại bỏ tất cả các thuộc tính liên quan đến nội dung nút. Trong các thử nghiệm của tác giả, chúng tôi sử dụng biến thể LINE sử dụng mức độ tương đồng bậc hai.
  \item $\mathbf{node2vec}$ [13] là một mô hình đồng nhất truyền thống và có thể coi là phiên bản tổng quát của DeepWalk [21]. Chúng tôi áp dụng mô hình này cho các đồ thị không đồng nhất theo cách tương tự như LINE.
  \item $\mathbf{ESim}$ [22] là một mô hình không đồng nhất truyền thống học cách biểu diễn nút từ các cấu hình metapath đã được lấy mẫu. ESim yêu cầu xác định trước các trọng số cho mỗi metapath. Ở đây, chúng tôi chỉ định các trọng số bằng nhau cho tất cả các metapath vì việc tìm kiếm  các trọng số tối ưu của các metapath là rất khó và không mang lại mức tăng hiệu suất đáng kể so với các trọng số mặc định bằng nhau theo các thử nghiệm của nhóm tác giả.
  \item $\mathbf{metapath2vec}$ [9] là một mô hình không đồng nhất truyền thống tạo ra các biểu diễn nút bằng cách cung cấp các random walks được định hướng bởi metapath cho một mô hình skip-gram. Mô hình này dựa trên một metapath do người dùng chỉ định, vì vậy chúng tôi thử nghiệm trên tất cả các metapath riêng biệt và tổng kết metapath có kết quả tốt nhất. Chúng tôi sử dụng biến thể metapath2vec++ trong các thử nghiệm của mình. 
  \item $\mathbf{HERec}$ [23] là một mô hình không đồng nhất truyền thống học cách biểu diễn nút bằng cách áp dụng DeepWalk cho các đồ thị đồng nhất dựa trên metapath được chuyển đổi từ đồ thị không đồng nhất ban đầu. Mô hình này đi kèm với một thuật toán kết hợp biểu diễn được thiết kế để dự đoán xếp hạng, có thể được điều chỉnh để dự đoán liên kết. Để phân loại/phân cụm nút, chúng tôi chọn và báo cáo kết quả cho metapath có hiệu suất tốt nhất.
  \item $\mathbf{GCN}$ [16] là một mô hình GNN đồng nhất. Mô hình này thực hiện các phép toán tích chập trong miền Fourier của đồ thị. Ở đây, chúng tôi kiểm tra hiệu suất của GCN trên các đồ thị đồng nhất dựa trên metapath và báo cáo kết quả cho metapath tốt nhất.
  \item $\mathbf{GAT}$ [28] là một GNN đồng nhất. Mô hình này thực hiện các thao tác tích chập trong miền không gian đồ thị với cơ chế kết hợp có chú ý. Tương tự, ở đây chúng tôi kiểm tra GAT trên các đồ thị đồng nhất dựa trên metapath và báo cáo kết quả cho metapath tốt nhất.
  \item $\mathbf{GATNE}$ [5] là một GNN không đồng nhất. Mô hình này tạo ra biểu diễn của nút từ biểu diễn cơ sở và biểu diễn cạnh, tập trung vào nhiệm vụ dự đoán liên kết. Ở đây chúng tôi báo cáo kết quả từ biến thể GATNE có hiệu quả tốt nhất.
  \item $\mathbf{HAN}$ [31] là một GNN không đồng nhất. Mô hình này học cách biểu diễn nút với metapath cụ thể từ các đồ thị đồng nhất khác nhau dựa trên metapath và tận dụng cơ chế chú ý để kết hợp chúng thành một biểu diễn vectơ cho mỗi nút.
\end{itemize}

Đối với các mô hình truyền thống, bao gồm LINE, node2vec, ESim, metapath2vec và HERec, chúng tôi thiết lập window size là 5, walk length là 100, số bước đi trên mỗi nút là 40 và số lượng mẫu âm tính là 5 (nếu có). Đối với các mô hình GNN (bao gồm GCN, GAT, HAN và MAGNN), chúng tôi đặt tỷ lệ dropout là 0,5; chúng tôi sử dụng tập dữ liệu huấn luyện, xác thực và kiểm tra có kích thước bằng nhau; chúng tôi sử dụng phương pháp Adam cho các nhiệm vụ tối ưu hóa với learning rate được thiết lập là 0,005 và tham số phân rã (L2) được đặt là 0,001; chúng tôi huấn luyện mô hình GNN trong 100 epochs và cho phép dừng sớm với ngưỡng patience là 30. Để phân loại nút và phân cụm nút, GNN được huấn luyện theo kiểu bán giám sát với một phần nhỏ các nút được gắn nhãn. Đối với GAT, HAN và MAGNN, chúng tôi thiết lập số lượng head chú ý là 8. Đối với HAN và MAGNN, chúng tôi thiết lập số chiều (thứ nguyên) của vectơ chú ý trong tập hợp giữa các metapath là 128. Để việc so sánh được công bằng, chúng tôi đặt số chiều (thứ nguyên) biểu diễn của tất cả các mô hình được đề cập ở trên đến 64.

\subsection{Phân loại nút (RQ1)}
Chúng tôi tiến hành thử nghiệm trên  tập dữ liệu IMDb và DBLP để so sánh hiệu năng của các mô hình khác nhau đối với tác vụ phân loại nút. Chúng tôi cung cấp biểu diễn của các nút được gắn nhãn (phim trong IMDb và tác giả trong DBLP) được tạo ra bởi mỗi mô hình cho bộ phân loại máy vectơ hỗ trợ tuyến tính (SVM) với các tỷ lệ đào tạo khác nhau. Lưu ý rằng để so sánh công bằng, chỉ các nút trong tập thử nghiệm được đưa vào SVM tuyến tính, bởi vì các mô hình bán giám sát đã "nhìn thấy" các nút trong tập huấn luyện và xác nhận, như được hiển thị trong Phương trình 11. Do đó, việc đào tạo và tỷ lệ thử nghiệm của SVM tuyến tính ở đây chỉ liên quan đến bộ thử nghiệm (nghĩa là 3478 nút cho IMDb và 3257 nút cho DBLP). Một lần nữa, các phân tách đào tạo/kiểm tra cho SVM tuyến tính cũng giống nhau trên các mô hình nhúng. Các chiến lược tương tự cũng được áp dụng cho các thí nghiệm về phân cụm nút và dự đoán liên kết. Chúng tôi báo cáo Macro-F1 và Micro-F1 trung bình của 10 lần chạy của từng mô hình nhúng trong Bảng 3.

We conduct experiments on the IMDb and DBLP datasets to compare the performance of different models on the node classification task. We feed the embeddings of labeled nodes (movies in IMDb and authors in DBLP) generated by each learning model to a linear support vector machine (SVM) classifier with varying training proportions. Note that for a fair comparison, only the nodes in the testing set are fed to the linear SVM, because semi-supervised models have already "seen" the nodes in the training and validation sets, as shown in Equation 11. Hence, the training and testing proportions of the linear SVM here only concern the testing set (i.e., 3478 nodes for IMDb and 3257 nodes for DBLP). Again, the train/test splits for the linear SVM are also the same across embedding models. Similar strategies are also applied to the experiments of node clustering and link prediction. We report the average Macro-F1 and Micro-F1 of 10 runs of each embedding model in Table 3.

As shown in the table, MAGNN performs consistently better than other baselines across different training proportions and datasets. On IMDb, it is interesting to see that node2vec performs better than traditional heterogeneous models. That said, GNNs, especially heterogeneous GNNs, obtain even better results, demonstrating that the GNN architecture, which judiciously utilizes the heterogeneous node features, helps improve the embedding performance. The performance gain obtained by MAGNN over the best baseline (HAN) is around 4-7\%, which indicates that metapath instances contain richer information than metapath-based neighbors. On DBLP, the node classification task is trivial, as evident from the high scores of all models. Even so, MAGNN still outperforms the strongest baseline by $1-2 \%$. 


\section{Kết luận}
Trong phạm vi báo cáo này, nhóm tác giả đề xuất mô hình MAGNN để giải quyết ba hạn chế đặc trưng của các phương pháp biểu diễn đồ thị không đồng nhất hiện nay là (1) loại bỏ các thuộc tính nội dung nút, (2) loại bỏ các nút trung gian dọc theo metapath và (3) chỉ xem xét một metapath duy nhất. Cụ thể, MAGNN được xây dựng dựa trên ba thành phần chính: (1) chuyển đổi nội dung nút, (2) tổng hợp intra-metapath và (3) tổng hợp inter-metapath để xử lý từng hạn chế tương ứng kể trên. Ngoài ra, chúng tôi xác định định nghĩa về mã hóa cấu hình metapath, từ đó có thể trích xuất thông tin cấu trúc và ngữ nghĩa ẩn sâu trong các cấu hình metapath. Chúng tôi đề xuất một số hàm mã hóa, trong đó bao gồm một hàm lấy ý tưởng từ mô hình biểu diễn đồ thị tri thức RotatE [24]. Trong các thử nghiệm, MAGNN đạt được kết quả tốt hơn trên cả ba bộ dữ liệu thực tế đối với các tác vụ phân loại nút, phân cụm nút và dự đoán liên kết. Nghiên cứu bóc tách ảnh hưởng của từng thành phần cũng chứng minh hiệu quả của ba thành phần chính của MAGNN trong việc tăng hiệu suất biểu diễn đồ thị. Các nghiên cứu tiếp theo có thể tập trung tìm kiếm các phương pháp áp dụng quy trình biểu diễn đồ thị này cho tác vụ dự đoán xếp hạng (hệ khuyến nghị) với dữ liệu người dùng-mặt hàng được thể hiện bởi các biểu đồ tri thức không đồng nhất [30].


\begin{thebibliography}{00}
\bibitem{b1} G. Eason, B. Noble, and I. N. Sneddon, ``On certain integrals of Lipschitz-Hankel type involving products of Bessel functions,'' Phil. Trans. Roy. Soc. London, vol. A247, pp. 529--551, April 1955.
\bibitem{b2} J. Clerk Maxwell, A Treatise on Electricity and Magnetism, 3rd ed., vol. 2. Oxford: Clarendon, 1892, pp.68--73.
\bibitem{b3} I. S. Jacobs and C. P. Bean, ``Fine particles, thin films and exchange anisotropy,'' in Magnetism, vol. III, G. T. Rado and H. Suhl, Eds. New York: Academic, 1963, pp. 271--350.
\bibitem{b4} K. Elissa, ``Title of paper if known,'' unpublished.
\bibitem{b5} R. Nicole, ``Title of paper with only first word capitalized,'' J. Name Stand. Abbrev., in press.
\bibitem{b6} Y. Yorozu, M. Hirano, K. Oka, and Y. Tagawa, ``Electron spectroscopy studies on magneto-optical media and plastic substrate interface,'' IEEE Transl. J. Magn. Japan, vol. 2, pp. 740--741, August 1987 [Digests 9th Annual Conf. Magnetics Japan, p. 301, 1982].
\bibitem{b7} M. Young, The Technical Writer's Handbook. Mill Valley, CA: University Science, 1989.
\end{thebibliography}
\vspace{12pt}
\color{red}
IEEE conference templates contain guidance text for composing and formatting conference papers. Please ensure that all template text is removed from your conference paper prior to submission to the conference. Failure to remove the template text from your paper may result in your paper not being published.

\end{document}
