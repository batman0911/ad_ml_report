Nhiều bộ dữ liệu thực tế được biểu diễn với cấu trúc dữ liệu đồ thị, trong đó các đối tượng và quan hệ giữa chúng được biểu diễn bằng các nút và cạnh. Các ví dụ bao gồm mạng xã hội [14, 29], hệ thống vật lý [2, 10], mạng giao thông [18, 34], mạng trích dẫn [1, 14, 16], hệ thống gợi ý [26, 35], đồ thị tri thức [3, 24], ... Bản chất non-Euclidean của đồ thị khiến chúng khó được mô hình hóa bằng các mô hình học máy truyền thống. Với tập hàng xóm của mỗi nút, không hề có thứ tự hoặc giới hạn về kích thước, Tuy nhiên, hầu hết các mô hình thống kê giả định rằng một đầu vào có thứ tự và kích thước cố định trong không gian Euclid. Do đó, sẽ thuận tiện nếu các nút có thể được biểu diễn bằng các vector thấp chiều trong không gian Euclid và từ đó có thể lấy làm đầu vào của mô hình học máy khác. 

Các kĩ thuật embed đồ thị khác nhau được đề xuất cho cấu trúc dữ liệu đồ thị. LINE [25] sinh node embedding dựa vào các nút gần nhất và gần thứ 2. Các phương pháp dựa trên bước ngẫu nhiên (Random-walk) bao gồm DeepWalk [21], node2vec [13] và TADW [32] sinh dãy nút được sinh ra bởi các bước ngẫu nhiên đến một mô hình skip-gram [19] để học node embeddings. Với sự phát triển nhanh chóng của deep learning, mạng neuron đồ thị (Graph neural networks - GNNs) được đề xuất, mô hình học các biểu diễn đồ thị bằng việc sử dụng các lớp neuron được thiết kế đặc biệt. Spectral-based GNNs bao gồm ChebNet [8] và GCN [16] biểu diễn các toán tử tích chập đồ thị trong miền Fourier của một đồ thị đầy đủ.