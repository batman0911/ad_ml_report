Trong phần này, tác giả xem lại các nghiên cứu về học biểu diễn đồ thị liên quan tới mô hình của tác giả. Chúng được chia thành hai phần nhỏ: phần đầu tóm tắt các nỗ lực nghiên cứu về GNN cho việc biểu diễn đồ thị, phần sau giới thiệu các phương pháp biểu diễn đồ thị cho các đồ thị không đông nhất.

\subsection{Mạng neuron đồ thị}
Mục tiêu của một GNN là học một biểu diễn vector thấp chiều $\pmb{h}_{\upsilon}$ cho mọi nút $\upsilon$, cách được sử dụng cho nhiều tác vụ downstream, ví dụ như phân loại nút, phân cụm nút và dự đoán liên kết. Lý do đằng sau điều này là mỗi nút được xác định một cách tự nhiên bởi các đặc trưng và khu vực lân cận của nó. Theo ý tưởng này và dựa trên quá trình xử lý tín hiệu đồ thị, GNNs dựa trên phổ được phát triển để thực hiện tích chập đồ thị trong miền Fourier của một đồ thị. ChebNet [8] sử dụng các đa thức Chebusev để lọc các tín hiệu đồ thị (đặc trưng nút) trong miền Fourier của đồ thị. Một mô hình có ảnh hưởng khác thuộc loại này là GCN [16], nó ràng buộc và đơn giản hóa các tham số của ChebNet để giảm bớt vấn đề overfitting và cải thiện hiệu năng. Tuy nhiên, GNNs dựa trên phổ có khả năng mở rộng và tổng quát kém vì chúng yêu cầu toàn bộ đồ thị làm đầu vào cho mọi lớp và các bộ lọc đã được học của chúng phụ thuộc vào cơ sở riêng của Laplacian của đồ thị, liên quan chặt chẽ đến cấu trúc đồ thị cụ thể.

GNNs dựa trên spatial đã được đề xuất để khắc phục những giới hạn này. GNNs kiểu này định nghĩa các tích chập một cách trực tiếp trong miền đồ thị bằng cách tổng hợp thông tin từ các lân cận của mỗi nút, giống như các toán tử tích chập trong mạng tích chập xử lý dữ liệu hình ảnh. GraphSAGE [14], một framework GNN dựa trên spatial được tạo ra dựa trên khái niệm chung về các hàm tổng hợp để tạo các nút nhúng hiệu quả. Các mẫu hàm tổng hợp và biến đổi một lân cận của nút mục tiêu và do đó tạo điều kiện huấn luyện song song và tổng quát hóa cho các nút hoặc đồ thị ẩn. Rất nhiều biến thể của GNN dựa trên spatial được đề xuất dựa trên ý tưởng này. Lấy cảm ứng từ Transformer [27], GAT [28] kết hợp cơ chế chú ý vào hàm tổng hợp để đưa vào độ quan trọng tương đối của thông tin của mỗ lân cận từ góc nhìn của nút đích. GGNN [17] thêm một thành phần lặp có kiểm soát (GRU) [7] vào hàm tổng hợp bằng cách xử lý thông tin vùng lân cận được tổng hợp làm đầu vào cho GRU của bước thời gian hiện tại. GaAN [34] kết hợp GRU với cơ chế chú ý nhiều đầu để  thỏa mãn không thời gian đồ thị. STAR-GCN [35] dử dụng nhiều bộ mã hóa-giải mã GCN để
tăng hiệu suất dự đoán xếp hạng. 

Tất cả các GNN được đề cập ở trên đều được xây dựng cho các đồ thị đồng nhất hoặc được thiết kế cho một số đồ thị với cấu trúc đặc biệt như là trong các hệ thống gợi ý người dùng - sản phẩm. Do hầu hết các GNNs hiện tại hoạt động trên đặc trưng của các nút trong cùng không gian nhúng được chia sẻ, chúng không thể đáp ứng một cách tự nhiên với các đồ thị không đồng nhất với các đặc trung nút nằm trên các không gian khác nhau.

\subsection[short]{Biểu diễn đồ thị không đồng nhất}
Biểu diễn đồ thị không đồng nhất nhằm mục đích chiếu các nút từ một đồ thị không đồng nhất vào một không gian vector thấp chiều. Chủ đề thách thức này thu hút rất nhiều nghiên cứu. Ví dụ, metapath2vec [9] sinh các bước ngẫu nhiên được hướng dẫn bởi một meta-path đơn, thứ sau đó được lấy làm đầu vào cho mô hình skip-gram [19] để sinh các nút nhúng. Với metapaths được định nghĩa trước, ESim [22] sinh các nút nhúng bằng cách học từ các cấu hình metapath mẫu âm và dương. HIN2vec [11] thực hiện nhiều tác vụ huấn luyện để học các biểu diễn của nút và metapaths của một đồ thị không đồng nhất. Cho một metapath, HERec [23] chuyển đổi một đồ thị không đồng nhất thành một đồ thị đồng nhất dựa trên các lân cận metapath-based và áp dụng mô hình DeepWalk  để học nhúng nút của các loại mục tiêu. Giống như HERec, HAN [31] chuyển đổi một đồ thị không đồng nhất thành nhiều đồ thị đồng nhất dựa trên metapath theo cách tương tự nhưng sử dụng một kiến trúc mạng chú ý đồ thị để tổng hợp thông tin từ các lân cận và thúc đẩy cơ chế chú ý để kết hợp nhiều metapaths. Một mô hình khá là PME [6] học nhúng nút bằng các chiếu chúng vào các không gian quan hệ tương ứng và tối ưu hóa tiệm cận giữa các nút được chiếu.

Tuy nhiên, tất cả các phương pháp biểu diễn đồ thị không đồng nhất được giới thiệu ở trên có những hạn chế là bỏ qua các đặc trưng về nội dung của nút, loại bỏ tất cả các nút trung gian dọc theo metapath hoặc chỉ sử dụng một metapath duy nhất. Mặc dù chúng có thể được cải thiện dựa trên hiệu năng của các phương pháp biểu diễn đồ thị không đồng nhất cho một số bộ dữ liệu đồ thị không đồng nhất, vẫn có thể làm tốt hơn bằng cách khai thác toàn diện hơn các thông tin trong đồ thị không đồng nhất. 