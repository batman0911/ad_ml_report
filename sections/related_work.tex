Phần này sẽ tập trung tổng kết về các phương pháp học cách biểu diễn đồ thị liên quan tới mô hình mà ta đang xét đến trong phạm vi báo cáo này. Trong đó gồm hai tiểu mục: phần đầu sẽ tóm tắt các công trình nghiên cứu về GNN để biểu diễn đồ thị, phần tiếp theo sẽ giới thiệu các phương pháp biểu diễn đồ thị cho các đồ thị không đồng nhất.

\subsection{Mạng neuron đồ thị}
Mục tiêu của một GNN là học cách biểu diễn vector thấp chiều $\pmb{h}_{\upsilon}$ cho mọi nút $\upsilon$, để từ đó có thể sử dụng cho nhiều bài toán tiếp theo sau đó, ví dụ như phân loại nút, phân cụm nút và dự đoán liên kết. Mục tiêu này là hoàn toàn hợp lí vì mỗi nút có thể được xác định một cách tự nhiên bởi các thuộc tính đặc trưng và khu vực lân cận của nó. Dựa trên ý tưởng này và quá trình xử lý tín hiệu đồ thị, các mô hình GNNs dựa trên phương pháp phổ (spectral-based) được phát triển để thực hiện tích chập đồ thị trong miền Fourier của một đồ thị. ChebNet [8] sử dụng các đa thức Chebusev để lọc các tín hiệu đồ thị (đặc trưng nút) trong miền Fourier của đồ thị. Một mô hình khác thuộc loại này cũng thường xuyên được nhắc đến chính là GCN [16], mô hình này ràng buộc và đơn giản hóa các tham số của ChebNet để giảm bớt vấn đề gây ra bởi hiện tượng overfitting và cải thiện hiệu năng tổng thể của mô hình. Tuy nhiên, GNNs dựa trên phổ có khả năng mở rộng và tổng quát hóa kém vì chúng yêu cầu sử dụng toàn bộ đồ thị làm đầu vào cho tất cả các lớp, đồng thời các bộ lọc (filters) cần được học của mô hình này lại phụ thuộc vào cơ sở riêng của Laplacian của đồ thị, liên quan chặt chẽ đến cấu trúc đồ thị cụ thể.

GNNs dựa trên không gian (spatial-based) đã được đề xuất để khắc phục những hạn chế kể trên. Các mô hình GNNs kiểu này xác định các tích chập một cách trực tiếp trong miền đồ thị bằng cách tổng hợp thông tin từ các lân cận của mỗi nút, giống như các toán tử tích chập trong mạng tích chập xử lý dữ liệu hình ảnh. GraphSAGE [14], một framework GNN dựa trên spatial được tạo ra dựa trên ý tưởng chung về các hàm tổng hợp để tạo các biểu diễn nút hiệu quả. Các hàm tổng hợp sẽ lấy mẫu, trích xuất, và biến đổi một lân cận của nút mục tiêu và từ đó thúc đẩy việc huấn luyện song song và tổng quát hóa cho các nút hoặc phần đồ thị chưa được quan sát (gán nhãn). Rất nhiều biến thể của GNN dựa trên không gian được đề xuất dựa trên ý tưởng này. Dựa trên ý tưởng từ Transformer [27], GAT [28] kết hợp cơ chế chú ý vào hàm tổng hợp để giải thích cho mức độ ảnh hưởng tương đối của một nút đích tới các lân cận. Trong khi đó, mô hình GGNN [17] thì lại thêm một thành phần lặp có kiểm soát (GRU) [7] vào hàm tổng hợp bằng cách xử lý thông tin vùng lân cận được tổng hợp làm đầu vào cho GRU của bước thời gian hiện tại. GaAN [34] kết hợp GRU với cơ chế chú ý nhiều đầu để thỏa mãn không thời gian đồ thị. STAR-GCN [35] dử dụng nhiều bộ mã hóa-giải mã GCN để tăng hiệu suất dự đoán xếp hạng. 

Tất cả các GNN được đề cập ở trên đều được xây dựng cho các đồ thị đồng nhất hoặc được thiết kế cho một số đồ thị với cấu trúc đặc biệt như là trong các hệ thống gợi ý người dùng - sản phẩm. Do hầu hết các mô hình GNNs hiện tại đều chỉ có thể hoạt động được trên các nút có thuộc tính đặc trưng trong cùng không gian biểu diễn chung, các mô hình này không thể hoạt động tương tự với các đồ thị không đồng nhất với các đặc trưng nút nằm trên các không gian khác nhau.

\subsection[short]{Biểu diễn đồ thị không đồng nhất}
Biểu diễn đồ thị không đồng nhất nhằm mục đích chiếu các nút từ một đồ thị không đồng nhất vào một không gian vector thấp chiều. Bài toán đầy thách thức này đã thu hút được sự quan tâm từ rất nhiều nghiên cứu. Ví dụ, metapath2vec [9] sinh các bước ngẫu nhiên được định hướng bởi một meta-path đơn, các bước ngẫu nhiên này sau đó được lấy làm đầu vào cho mô hình skip-gram [19] để sinh ra các biểu diễn nút. Với metapaths được định nghĩa trước, ESim [22] sinh các biểu diễn nút bằng cách học từ các cấu hình metapath đã được lấy mẫu và gán nhãn negative và positive. HIN2vec [11] thực hiện nhiều tác vụ huấn luyện để học các biểu diễn của nút và metapaths của một đồ thị không đồng nhất. Cho một metapath, HERec [23] chuyển đổi một đồ thị không đồng nhất thành một đồ thị đồng nhất dựa trên các lân cận metapath-based và áp dụng mô hình DeepWalk để học biểu diễn nút của các loại mục tiêu. Giống như HERec, HAN [31] chuyển đổi một đồ thị không đồng nhất thành nhiều đồ thị đồng nhất dựa trên metapath theo cách tương tự nhưng sử dụng một kiến trúc mạng đồ thị có chú ý để tổng hợp thông tin từ các lân cận và thúc đẩy cơ chế chú ý để kết hợp nhiều metapaths. Một mô hình khác là PME [6] học cách biểu diễn nút bằng cách chiếu chúng vào các không gian quan hệ tương ứng và tối ưu hóa mức độ tương đồng giữa các nút được chiếu.

Tuy nhiên, tất cả các phương pháp biểu diễn đồ thị không đồng nhất được giới thiệu ở trên có những hạn chế là bỏ qua các đặc trưng về nội dung của nút, loại bỏ tất cả các nút trung gian dọc theo metapath hoặc chỉ sử dụng một metapath duy nhất. Mặc dù chúng có thể được cải thiện dựa trên hiệu năng của các phương pháp biểu diễn đồ thị không đồng nhất cho một số bộ dữ liệu đồ thị không đồng nhất, ta vẫn có thể làm tốt hơn bằng cách khai thác toàn diện hơn các thông tin trong đồ thị không đồng nhất. 
