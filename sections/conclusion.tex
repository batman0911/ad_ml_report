Trong phạm vi báo cáo này, nhóm tác giả đề xuất mô hình MAGNN để giải quyết ba hạn chế đặc trưng của các phương pháp biểu diễn đồ thị không đồng nhất hiện nay là (1) loại bỏ các thuộc tính nội dung nút, (2) loại bỏ các nút trung gian dọc theo metapath và (3) chỉ xem xét một metapath duy nhất. Cụ thể, MAGNN được xây dựng dựa trên ba thành phần chính: (1) chuyển đổi nội dung nút, (2) tổng hợp intra-metapath và (3) tổng hợp inter-metapath để xử lý từng hạn chế tương ứng kể trên. Ngoài ra, tác giả xác định định nghĩa về mã hóa cấu hình metapath, từ đó có thể trích xuất thông tin cấu trúc và ngữ nghĩa ẩn sâu trong các cấu hình metapath. tác giả đề xuất một số hàm mã hóa, trong đó bao gồm một hàm lấy ý tưởng từ mô hình biểu diễn đồ thị tri thức RotatE [24]. Trong các thử nghiệm, MAGNN đạt được kết quả tốt hơn trên cả ba bộ dữ liệu thực tế đối với các tác vụ phân loại nút, phân cụm nút và dự đoán liên kết. Nghiên cứu bóc tách ảnh hưởng của từng thành phần cũng chứng minh hiệu quả của ba thành phần chính của MAGNN trong việc tăng hiệu suất biểu diễn đồ thị. Các nghiên cứu tiếp theo có thể tập trung tìm kiếm các phương pháp áp dụng quy trình biểu diễn đồ thị này cho tác vụ dự đoán xếp hạng (hệ khuyến nghị) với dữ liệu người dùng-mặt hàng được thể hiện bởi các biểu đồ tri thức không đồng nhất [30].
